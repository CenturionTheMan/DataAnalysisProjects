Przeprowadzone badania wykazały, że barwa oprawek okularów ochronnych ma istotne znaczenie na czas skupienia wzroku na żółtej torbie. 
Uczestnicy badania, którzy nosili okulary z oprawkami w kolorze żółtym, 
spędzali średnio 0.47 sekundy krócej na obserwacji żółtej torby niż uczestnicy, którzy nosili okulary z oprawkami przezroczystymi i 
0.39 sekundy krócej niż uczestnicy, którzy nosili okulary z oprawkami czerwonymi (tabela [\ref{tab:summaryTFD_yBag}]). 
Otrzymane wyniki sugerują, że kolor oprawek okularów ochronnych może wpływać na zdolność do dostrzegania obiektów o podobnym kolorze.
Wyniki te mogą mieć istotne znaczenie w kontekście bezpieczeństwa pracy.

Dodatkowa hipoteza zakładająca, że doświadczenie zawodowe uczestników ma wpływ na czas skupienia wzroku na żółtej torbie,
nie została potwierdzona. 
Nie stwierdzono istotnych różnic w czasie skupienia wzroku na żółtej torbie pomiędzy uczestnikami z doświadczeniem zawodowym a tymi bez doświadczenia (tabela [\ref{tab:summaryExp_ybag}]).