Niniejszy raport analizuję wybrane aspekty zbioru danych 
\textit{''Recording visual behaviors of individuals wearing goggles with yellow, red, and transparent frames during H\&S inspections of situations presented in videos''} 
\cite{4W6TIW_2025} opracowanego w 2024 roku w laboratorium LET's GO LAB Politechniki Wrocławskiej. Badanie miało na celu ustalić,
czy barwa oprawek okularów ochronnych wpływa na widoczność obiektów o różnych kolorach. 

Badanie składało się z kilku etapów:

\begin{enumerate}
    \item Pierwszy krok polegał na wypełnieniu kwestionariuszy z danymi demograficznymi.
    \item Następnie badany był wzrok uczestników - umiejętność rozróżniania kolorów.
    \item Kolejno uczestnicy wypełniali test składający się z 10 pytań o zdrowiu i bezpieczeństwie (\textit{''Health and Safety''}). Uczestnicy byli poinformowani, że badana będzie umiejętność wykrywania niebezpiecznych sytuacji w filmach.
    \item W następnym kroku uczestnicy oglądali dwa filmy, w których mieli wykrywać niebezpieczne sytuacje. Każdy uczestnik miał założone okulary z oprawkami w kolorze żółtym, czerwonym lub przezroczystym. Podczas drugiego filmu rejestrowany był ruch gałek ocznych uczestników.
    \item Na koniec uczestnikom zdradzany był prawdziwy cel badania.
\end{enumerate}

W nagraniu pojawiało się 6 kolorowych obiektów:
\begin{itemize}
    \item czerwone wiadro,
    \item czerwony hełm,
    \item czerwona kurtka,
    \item żółta torba,
    \item żółte wiadro,
    \item żółty hełm
\end{itemize}

Dla każdego obiektu rejestrowane były dwie zmienne:
\begin{itemize}
    \item TFD - (\textit{Total Fixation Duration}) - całkowity czas fiksacji wzroku na obiekcie.
    \item TTFF - (\textit{Time to First Fixation}) - czas do pierwszej fiksacji wzroku na obiekcie.
\end{itemize}

W ramach niniejszego raportu skoncentrowano się na jednym, wyodrębnionym aspekcie badania: 
wpływie koloru gogli na długość obserwacji żółtej torby (\textit{TFD-Y bag}) przez uczestników badania.