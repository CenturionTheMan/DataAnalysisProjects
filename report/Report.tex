\documentclass{article}
\nocite{*}


\input{TexBase/DocumentBase.tex}


\section{Wprowadzenie} %TODO
\section{Uczestnicy badania}
    \subsection{Analiza deskryptywna zmiennych demograficznych}
        \subsubsection*{Wiek}
        \begin{table}[H]
            \centering
            \caption{Opis deskryptywny wieku uczestników badania.}
            \begin{tabular}{|c|c|c|c|}%
                \hline
                \bfseries Miara & \bfseries Gogle przezroczyste & \bfseries Gogle czerwone & \bfseries Gogle żółte% specify table head
                \csvreader[head to column names]{./../res_tables/summaryAge.csv}{}% use head of csv as column names
                {\\\hline\Miara & \num{\T} & \num{\R} & \num{\Y}}% specify your columns here
                \\\hline    
            \end{tabular}
            \label{tab:summaryAge}
        \end{table}

        \begin{figure}[H]
            \centering
            \caption{Histogram dla wieku uczestników badania.}
            \resizebox{0.7 \columnwidth}{!}{%
            \includegraphics{./../res_plots/Histogram_dla_wieku.png}%
            }
            \label{fig:histAge}
        \end{figure}

        \begin{figure}[H]
            \centering
            \caption{Wykres pudełkowy dla wieku uczestników badania.}
            \resizebox{0.7 \columnwidth}{!}{%
            \includegraphics{./../res_plots/Wykres_pudełkowy_dla_wieku.png}%
            }
            \label{fig:boxAge}
        \end{figure}

        \subsubsection*{Doświadczenie zawodowe}
        \begin{table}[H]
            \centering
            \caption{Opis deskryptywny doświadczenia zawodowego uczestników badania.}
            \begin{tabular}{|c|c|c|c|}%
                \hline
                \bfseries Miara & \bfseries Gogle przezroczyste & \bfseries Gogle czerwone & \bfseries Gogle żółte% specify table head
                \csvreader[head to column names]{./../res_tables/summaryExperience.csv}{}% use head of csv as column names
                {\\\hline\Miara & \num{\T} & \num{\R} & \num{\Y}}% specify your columns here
                \\\hline    
            \end{tabular}
            \label{tab:summaryExperience}
        \end{table}

        \begin{figure}[H]
            \centering
            \caption{Histogram dla doświadczenia zawodowego uczestników badania.}
            \resizebox{0.7 \columnwidth}{!}{%
            \includegraphics{./../res_plots/Histogram_dla_doświadczenia_zawodowego.png}%
            }
            \label{fig:histExperience}
        \end{figure}


        \subsubsection*{Wyniki testu \textit{''health and safety''} (H\&S)}
        \begin{table}[H]
            \centering
            \caption{Opis deskryptywny wyników testu \textit{''health and safety''} (H\&S) uczestników badania.}
            \begin{tabular}{|c|c|c|c|}%
                \hline
                \bfseries Miara & \bfseries Gogle przezroczyste & \bfseries Gogle czerwone & \bfseries Gogle żółte% specify table head
                \csvreader[head to column names]{./../res_tables/summaryHSTestResults.csv}{}% use head of csv as column names
                {\\\hline\Miara & \num{\T} & \num{\R} & \num{\Y}}% specify your columns here
                \\\hline    
            \end{tabular}
            \label{tab:summaryHSTestResults}
        \end{table}

        \begin{figure}[H]
            \centering
            \caption{Histogram dla wyników testu \textit{''health and safety''} (H\&S) uczestników badania.}
            \resizebox{0.7 \columnwidth}{!}{%
            \includegraphics{./../res_plots/Histogram_dla_wyników_testu_H&S.png}%
            }
            \label{fig:histHSTestResults}
        \end{figure}

        \begin{figure}[H]
            \centering
            \caption{Wykres pudełkowy dla wyników testu \textit{''health and safety''} (H\&S) uczestników badania.}
            \resizebox{0.7 \columnwidth}{!}{%
            \includegraphics{./../res_plots/Wykres_pudełkowy_dla_wyników_testu_H&S.png}%
            }
            \label{fig:boxHSTestResults}
        \end{figure}

        \subsubsection*{Płeć}
        \begin{table}[H]
            \centering
            \caption{Opis deskryptywny płci uczestników badania.}
            \begin{tabular}{|c|c|c|c|}%
                \hline
                \bfseries Miara & \bfseries Gogle przezroczyste & \bfseries Gogle czerwone & \bfseries Gogle żółte% specify table head
                \csvreader[head to column names]{./../res_tables/summarySex.csv}{}% use head of csv as column names
                {\\\hline\Miara & \num{\T} & \num{\R} & \num{\Y}}% specify your columns here
                \\\hline    
            \end{tabular}
            \label{tab:summarySex}
        \end{table}

        \begin{figure}[H]
            \centering
            \caption{Histogram dla płci uczestników badania.}
            \resizebox{0.7 \columnwidth}{!}{%
            \includegraphics{./../res_plots/Histogram_dla_płci_(F=1,_M=2,_O=3).png}%
            }
            \label{fig:histEsx}
        \end{figure}


    \subsection{Czas noszenia gogli}
    \begin{table}[H]
        \centering
        \caption{Opis deskryptywny czasu noszenia gogli uczestników badania.}
        \begin{tabular}{|c|c|c|c|}%
            \hline
            \bfseries Miara & \bfseries Gogle przezroczyste & \bfseries Gogle czerwone & \bfseries Gogle żółte% specify table head
            \csvreader[head to column names]{./../res_tables/summaryTime.csv}{}% use head of csv as column names
            {\\\hline\Miara & \num{\T} & \num{\R} & \num{\Y}}% specify your columns here
            \\\hline    
        \end{tabular}
        \label{tab:summaryTime}
    \end{table}

    \begin{figure}[H]
        \centering
        \caption{Histogram dla czasu noszenia gogli uczestników badania.}
        \resizebox{0.7 \columnwidth}{!}{%
        \includegraphics{./../res_plots/Histogram_dla_czasu.png}%
        }
        \label{fig:histTime}
    \end{figure}

    \begin{figure}[H]
        \centering
        \caption{Wykres pudełkowy dla czasu noszenia gogli uczestników badania.}
        \resizebox{0.7 \columnwidth}{!}{%
        \includegraphics{./../res_plots/Wykres_pudełkowy_dla_czasu.png}%
        }
        \label{fig:boxTime}
    \end{figure}

\section{Analiza TFD dla obiektu żółta torba (yellow bag)} %TODO
    \subsection{Hipotezy}
    \begin{itemize}
        \item $H_0$: Nie ma różnicy w czasie skupienia na żółtej torbie (TFD-Y bag) pomiędzy grupami z doświadczeniem i bez doświadczenia.
        \item $H_1$: Grupa z doświadczaniem ma dłuższy czas skupienia na żółtej torbie (TFD-Y bag) niż grupa bez doświadczenia.
    \end{itemize}
    \subsection{Analiza deskryptywna zmiennej}
    \begin{table}[H]
        \centering
        \caption{Opis deskryptywny zmiennej TFD dla obiektu żółta torba (yellow bag).}
        \begin{tabular}{|c|c|c|c|}%
            \hline
            \bfseries Miara & \bfseries Gogle przezroczyste & \bfseries Gogle czerwone & \bfseries Gogle żółte% specify table head
            \csvreader[head to column names]{./../res_tables/summaryTFD_yBag.csv}{}% use head of csv as column names
            {\\\hline\Miara & \num{\T} & \num{\R} & \num{\Y}}% specify your columns here
            \\\hline    
        \end{tabular}
        \label{tab:summaryTFD_yBag}
    \end{table}
    % \begin{figure}[H]
    %     \centering
    %     \caption{Histogram dla zmiennej TFD dla obiektu żółta torba (yellow bag).}
    %     \resizebox{0.7 \columnwidth}{!}{%
    %     \includegraphics{./../res_plots/Histogram_dla_czasu_skupienia_na_żółtej_torbie.png}%
    %     }
    %     \label{fig:histTFD_yBag}
    % \end{figure}
    \begin{figure}[H]
        \centering
        \caption{Wykres pudełkowy dla zmiennej TFD dla obiektu żółta torba (yellow bag).}
        \resizebox{0.7 \columnwidth}{!}{%
        \includegraphics{./../res_plots/Wykres_pudełkowy_dla_czasu_skupienia_na_żółtej_torbie.png}%
        }
        \label{fig:boxTFD_yBag}
    \end{figure}

    \subsection{Równoliczność grup} %TODO
    Dla sprawdzenia równoliczności grup wykonano test $\chi^2$.
    Hipotezy:
    \begin{itemize}
        \item $H_0$: liczebności grup są równe.
        \item $H_1$: liczebności grup są różne.
    \end{itemize}
    Otrzymana wartość wynosi $p=0.926$, co oznacza, że na poziomie istotności $\alpha=0.05$
    nie ma podstaw do odrzucenia hipotezy zerowej. 
    
    \subsection{Normalność zmiennej w grupach}
    \begin{table}[H]
        \centering
        \caption{Wyniki testu Shapiro-Wilka dla czasu skupienia na żółtej torbie (bez przekształceń).}
        \begin{tabular}{|c|c|c|}%
            \hline
            \bfseries Kolor okularów & \bfseries Wartość p & \bfseries Czy rozkład normalny% specify table head
            \csvreader[head to column names]{./../res_tables/yBag_shapiro_default.csv}{}% use head of csv as column names
            {\\\hline\kolorGogli & \num{\shapiroP} & \czyNormalny}% specify your columns here
            \\\hline    
        \end{tabular}
        \label{tab:shapiroYBagDefault}
    \end{table}
    \begin{figure}[H]
        \centering
        \caption{Histogram dla czasu skupienia na żółtej torbie (bez przekształceń).}
        \resizebox{0.7 \columnwidth}{!}{%
        \includegraphics{./../res_plots/historgram_dla_TFD_yBag_default.png}%
        }
        \label{fig:histYBagDefault}
    \end{figure}

    \begin{table}[H]
        \centering
        \caption{Wyniki testu Shapiro-Wilka dla czasu skupienia na żółtej torbie (pierwiastek z X).}
        \begin{tabular}{|c|c|c|}%
            \hline
            \bfseries Kolor okularów & \bfseries Wartość p & \bfseries Czy rozkład normalny% specify table head
            \csvreader[head to column names]{./../res_tables/yBag_shapiro_x^0.5.csv}{}% use head of csv as column names
            {\\\hline\kolorGogli & \num{\shapiroP} & \czyNormalny}% specify your columns here
            \\\hline    
        \end{tabular}
        \label{tab:shapiroYBagPow0.5}
    \end{table}
    \begin{figure}[H]
        \centering
        \caption{Histogram dla czasu skupienia na żółtej torbie (pierwiastek z X).}
        \resizebox{0.7 \columnwidth}{!}{%
        \includegraphics{./../res_plots/historgram_dla_TFD_yBag_x^0.5.png}%
        }
        \label{fig:histYBagPow0.5}
    \end{figure}

    \begin{table}[H]
        \centering
        \caption{Wyniki testu Shapiro-Wilka dla czasu skupienia na żółtej torbie (pierwiastek kwadratowy z X).}
        \begin{tabular}{|c|c|c|}%
            \hline
            \bfseries Kolor okularów & \bfseries Wartość p & \bfseries Czy rozkład normalny% specify table head
            \csvreader[head to column names]{./../res_tables/yBag_shapiro_x^0.25.csv}{}% use head of csv as column names
            {\\\hline\kolorGogli & \num{\shapiroP} & \czyNormalny}% specify your columns here
            \\\hline    
        \end{tabular}
        \label{tab:shapiroYBagPow0.25}
    \end{table}
    \begin{figure}[H]
        \centering
        \caption{Histogram dla czasu skupienia na żółtej torbie (pierwiastek kwadratowy z X).}
        \resizebox{0.7 \columnwidth}{!}{%
        \includegraphics{./../res_plots/historgram_dla_TFD_yBag_x^0.25.png}%
        }
        \label{fig:histYBagPow0.25}
    \end{figure}

    \begin{table}[H]
        \centering
        \caption{Wyniki testu Shapiro-Wilka dla czasu skupienia na żółtej torbie (logarytm z X).}
        \begin{tabular}{|c|c|c|}%
            \hline
            \bfseries Kolor okularów & \bfseries Wartość p & \bfseries Czy rozkład normalny% specify table head
            \csvreader[head to column names]{./../res_tables/yBag_shapiro_log(x).csv}{}% use head of csv as column names
            {\\\hline\kolorGogli & \num{\shapiroP} & \czyNormalny}% specify your columns here
            \\\hline    
        \end{tabular}
        \label{tab:shapiroYBagLog}
    \end{table}
    \begin{figure}[H]
        \centering
        \caption{Histogram dla czasu skupienia na żółtej torbie (logarytm z X).}
        \resizebox{0.7 \columnwidth}{!}{%
        \includegraphics{./../res_plots/historgram_dla_TFD_yBag_log(x).png}%
        }
        \label{fig:histYBagLog}
    \end{figure}

    \begin{figure}[H]
        \centering
        \caption{Wykres kwartyl-kwartyl dla czasu skupienia na żółtej torbie }
        \resizebox{0.7 \columnwidth}{!}{%
        \includegraphics{./../res_plots/qqplot_dla_TFD_yBag.png}%
        }
        \label{fig:qqplotYBagLog}
    \end{figure}

    Dla grupy czerwonej i przezroczystej wykonane testy wskazują na normalność rozkładu zmiennej TFD-Y bag.
    Dla grupy żółtej testy shapiro-wilka nie pozwala na przyjcie H0 świadczącej o normalności rozkładu,
    niemniej otrzymana wartość p jest bliska granicy istotności $\alpha=0.05$ oraz wykresy histogramu i wykresy
    kwartyl-kwartyl sugerują, że rozkład jest zbliżony do normalnego.
    Dlatego można przyjąć, że rozkład zmiennej TFD-Y bag jest normalny we wszystkich grupach. 
    \subsection{Równość wariancji w grupach}
        Na podstawie wyników z sekcji \textit{''Normalność zmiennej w grupach''} stwierdzono, 
        normalność rozkładu zmiennej TFD-Y bag w grupach. Mając jednak na uwadze, że rozkład
        zmiennej TFD-Y bag w grupie żółtej nie jest idealnie normalny, przeprowadzono test Levene'a
        wycentrowanego na podstawie średniej.

        Hipotezy testowe:
        \begin{itemize}
            \item $H_0$: wariancje we wszystkich grupach są równe.
            \item $H_1$:  co najmniej jedna grupa ma inną wariancję.
        \end{itemize}
        Otrzymana wartość wynosi $p=0.240$, co oznacza, że na poziomie istotności $\alpha=0.05$
        nie ma podstaw do odrzucenia hipotezy zerowej.

    \subsection{Równość średnich w grupach} %TODO
        \subsubsection{Uzasadnienie wyboru testu na podstawie wyników analiz z punktów 2-5} %TODO
        \subsubsection{Przeprowadzenie testu - wynik i wnioski} %TODO
    \subsection{Wpływ doświadczenia na zmienną TFD} %TODO
\section{Analiza wrażliwości} %TODO
\section{Wnioski i podsumowanie} %TODO

\end{document}